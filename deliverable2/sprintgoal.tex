\section{Sprint Goal}
% describe your experiences with version control and continuous integration

We all had prior experience with Github. However the experience was
mostly with everyone comitting to the master branch. For this project we
chose to create branches in the repositories. We also chose to maintain to
repositories, one for the project reports and experimental code, and one
for the production code, containing the entire Leapkit solution.

The experimental repository contains 3 branches, \texttt{Master}, which
is the main branch, it must be clean and all code/LaTeX must always be
compileable. Commits directly to this branch is not allowed. We also have
the \texttt{report} branch which is used for working with the reports. The
last branch is \texttt{LinkedIn}, this branch contains experimental code
for LinkedIn integration and notes for how to work with the data.

The second repository is the production repository, this contains
the entire Leapkit solution. Apart from the \texttt{master}
branch which contains final releases. It also contains the branch
\texttt{linkedin} - for work with the LinkedIn integration, and the branch
\texttt{matchmaking} which is used for the algorithm development.

For both repositories the only way to add code to \texttt{master} is
to create pull requests from the relevant branches. To implement the
continous integration strategy we will perform pull requests regularly
to make sure \texttt{master} is always up to date. But pull requests are
only approved if the branch is runnable and pass any relevant test. 
\todo[inline]{Add experiences about how this worked out.}

We are setting up Jenkins and a test suite. The test suite shall seek to follow the Python standard on testing: \url{http://docs.python-guide.org/en/latest/writing/tests/}, using \textit{unittest} and \textit{pytest}.