\section{Sprint Retrospective}
% what went well, what went less well, and what will you improve

We established a good connection to our project partners and had a
couple of meetings where we met everyone in the company. We have firmly
established what the project is about and the scope of the project.

We managed to complete almost all of our original sprint goals (and added some later), but the work load was a bit lighter than intended. So we did a descent estimation of the time score of the backlog items handled in this sprint.

Jenkins got set up after we started programming and is not fully ready for use, but by the start of the next sprint, we expect to have it up and running. This was due to underestimating the time to setup the virtual machine with the Leapkit environment, which should be included in the CI environmet. We do however have Jenkins setup to handle our (at the moment) seperate python code, but we still need to develop a rigid test suite.

The Scrum Master did not assign tasks to individual people in the
group. Instead people assigned themselves to the tasks they felt like
doing. In the future we will seek to make this assignment in the
initial sprint meeting and adjust these at the workdaily Scrum meetings,
such that the the workload is better balanced between the group members.

People have been bad at meeting at the agreed time, so we will try to improve this. The fairly noticeable difference in peoples' arrival time have caused our work schedules to become rather ad hoc and our workdaily Scrum meetings to be later on the day instead of the initial task (for those who arrive on time). This will definitely need to be changed in the upcoming sprint.
